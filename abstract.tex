\begin{abstract}
    The current work presents a comprehensive study on the spectral characteristics and parameters of the supersoft X-ray source \source. We utilize a dataset comprising of six observations conducted by four different space observatories, namely ASCA, Chandra, XMM-Newton and NICER, spanning a time period of 25 years. Our objective is to identify a robust NLTE spectral model that can provide an acceptable fit to the continuum spectrum of \source, which had not been performed earlier as per current literature, thereby enhancing our understanding of its astrophysical nature. The study employs rigorous spectral modeling and comparative analysis to establish a framework for characterizing the continuum spectrum of \source. The study also discusses the analysis of the supersoft spectrum of \source, including the challenges in reproducing its spectra using NLTE model atmospheres. The best-fit model consists of a pure hydrogen NLTE model with effective gravity of 7, in conjunction with photoelectric absorption, ISM absorption and edge components. The effective temperature is found to be approximately of the order of $\sim 10^5$ K for all six observations, suggesting the presence of a hot accretion disk surrounding the white dwarf, which accretes matter from a companion main-sequence star. The study also analyzes the relative strengths of absorption edges in the spectrum and notes inconsistencies in the NICER observations. {The current work demonstrates that while the continuum spectrum of \source\ can be fitted satisfactorily, the same cannot be said to be the case for its high-resolution grating spectra. Therefore, it makes a case for the need to explore and develop methods that study such latest grating spectra, so as to obtain a more nuanced view of the astrophysics involving \source.}
\end{abstract}

\keywords{X-rays: binaries, X-rays: individual: \source, methods: observational, techniques: spectroscopic}
% https://www.inaoep.mx/~ydm/aj_keywords.html
