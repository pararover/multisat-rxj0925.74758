    \subsection{NLTE pure H model atmosphere}
    In a classical stellar atmosphere, which is a plane-parallel, horizontally homogenous atmosphere in hydrostatic and radiative equilibrium, a non-LTE (or NLTE) description refers to a scenario where the energy levels of some selected species may be allowed to depart from their local thermodynamic equilibrium (LTE) values \cite{hubeny2014theory}.
    	\subsubsection{Structural equations}
    	\quad\\	
	    Given below are the structural equations that are used to model an NLTE atmosphere.
	   
		In its second-order form, the \textit{radiative transfer equation} is
		\begin{align}
			\npdvt{2}{(f_\nu J_\nu)}{\tau_\nu}=J_\nu-S_\nu \label{eqn:rte-2nd-order}
		\end{align}
		where $J_\nu$ is the \textit{mean intensity} over all solid angles, $\tau_\nu$ the monochromatic \textit{optical depth}, $f_\nu$ the variable \textit{Eddington factor} and $S_\nu$ the \textit{source function}. The upper and lower boundary conditions on equation (\ref{eqn:rte-2nd-order}) are respectively
		\begin{align}
			&\left[ \pdvt{(f_\nu J_\nu)}{\tau_\nu} \right]_0=g_\nu J_\nu(0)-H_\nu^\text{ext} \label{eqn:rte-upper-bc} \\
			&\left[ \pdvt{(f_\nu J_\nu)}{\tau_\nu} \right]_{\tau_\text{max}}=H_\nu^+-\dfrac{1}{2}J_\nu \label{eqn:rte-lower-bc}
		\end{align}
		where $g_\nu$ is the \textit{surface Eddington factor}, $H_\nu^\text{ext}$, $H_\nu^+$ being the \textit{first angular moment} of specific intensity taken at the respective upper and lower boundaries of the current stellar atmospheric layer.
		
		The \textit{hydrostatic equilibrium equation} is re-cast as
		\begin{align}
			\dvt{P_\text{gas}}{m}=g-\dfrac{4\pi}{c}\integ{0}{\infty}{\dvt{K_\nu}{m}}{\nu}=g-\dfrac{4\pi}{c}\integ{0}{\infty}{\dfrac{\chi_\nu}{\rho}H_\nu}{\nu} \label{eqn:hyd-equil}
		\end{align}
		where $K_\nu$ is the \textit{second angular moment} of specific intensity, $\chi_\nu$ is the \textit{absorption coefficient} and $m$ is the \textit{Lagrangian mass}. In equation (\ref{eqn:hyd-equil}), the total pressure is composed of three parts: the \textit{gas pressure} $P_\text{gas}$, the \textit{radiation pressure} represented by the integrals on the right-hand side and the \textit{microturbulent pressure} $P_\text{turb}$ being ignored.
		
		The \textit{radiative equilibrium equation} is an expression of the conservation of total radiant flux. It may be written in an \textit{integral form} -- more accurate at upper atmospheric layer, or in a \textit{differential form} -- more accurate in deeper layers. To improve the accuracy, numerical algorithms implement a linear combination of both form as
		\begin{align}
			\alpha&\left[ \integ{0}{\infty}{(\kappa_\nu J_\nu-\eta_\nu)}{\nu} \right] \nonumber\\
				&+\beta\left[ \integ{0}{\infty}{\dvt{(f_\nu J_\nu)}{\tau_\nu}}{\nu}-\dfrac{\sigma_R}{4\pi}T_\text{eff}^4 \right]=0 \label{eqn:rad-equil}
		\end{align}
		where $\sigma_R$ is the Stefan-Boltzmann constant, $\kappa_\nu$, $\eta_\nu$ are the thermal absorption and emission coefficients respectively. In equation (\ref{eqn:rad-equil}) above, the term within the first pair of brackets contains the integral form and that within the second pair contains the differential form. The empirical coefficients $\alpha$, $\beta$ enable a transition from upper layers ($\alpha\rightarrow 1$, $\beta\rightarrow 0$) to lower layers ($\alpha\rightarrow 0$, $\beta\rightarrow 1$) -- such a transition is taken to be around the depth where the Rosseland mean optical depth is $\sim 1$.
		
		The \textit{kinetic equilibrium equations} (or \textit{rate equations}) are written for each chemical species, after having first selected those species which are to be considered to deviate from LTE. For any species, these equations may be represented by a matrix equation as
		\begin{align}
			\vsmat{A}\cdot\vsvec{n}=\vsvec{b} \label{eqn:kin-equil}
		\end{align}
		where the elements of the \textit{rate matrix} $\vsmat{A}$ are given as
		\begin{align*}
			\vmelem{A}{i}{j}&=\sum_{j\neq i}{(R_{ij}+C_{ij})} \\
			\vmelem{A}{i}{j}&=-(R_{ji}+C_{ji}),\text{ for }j\neq i\text{ and }i\neq k \\
			\vmelem{A}{k}{j}&=1+S_j
		\end{align*}
		with $k$ being the index of the \textit{characteristic level} and $R_{ij}$, $C_{ij}$ the radiative and collisional rates respectively between transition levels $i$ and $j$. It must be noted that $R_{ij}$'s implicitly involve the \textit{Saha-Boltzmann factor} $\Phi_i(T)$ which comes into play for \textit{bound-free transitions}, i.e. absorption edges
		\begin{align}
			\Phi_i(T)=\dfrac{g_i}{2g_1^+}\left( \dfrac{h^2}{2\pi m_e kT} \right)^{3/2}e^{(E_I-E_i)/kT} \label{eqn:saha-boltz-factor}
		\end{align}
		where $E_I$ is the ionization potential of the ion to which level $i$ belongs, $E_i$ the excitation energy of level $i$ and $g_1^+$ the statistical weight of the ground state of the next ion and $m_e$ the electronic mass. In equation (\ref{eqn:kin-equil}), $\vsvec{n}$ is the vector of populations of levels and the vector $\vsvec{b}$ has a single non-zero element corresponding to the characteristic level $k$ as
		\begin{align*}
			\vvelem{b}{i}=(N-n_e)\alpha_I\delta_{ki}
		\end{align*}
		where $N$ is the net populations of all levels, $n_e$ the number density of electrons and $\alpha_I$ the fractional abundance of the species $I$.
		
		The overall electrical neutrality of the medium is expressed by the \textit{charge conservation equation} as
		\begin{align}
			\sum_i{n_iZ_i}-n_e=0 \label{eqn:ch-cons}
		\end{align}
		where $Z_i$ is the charge associated with level $i$.
		
		The \textit{mass density}, which is related to the atomic level populations, can be expressed as
		\begin{align}
			\rho=(N-n_e)\mu m_H \label{eqn:mass-dens}
		\end{align}
		where $\mu$ is the mean molecular weight and $m_H$ is the mass of the H atom. A quantity known as the \textit{fictitious massive particle density} is defined as
		\begin{align}
			n_m\equiv(N-n_e)\mu \label{eqn:fict-mass-dens}
		\end{align}
		which then enables the mass density to be simply written as $\rho=n_m m_H$.
		
		The absorption coefficient included in equation (\ref{eqn:hyd-equil}) is written as the combination
		\begin{align}
			\chi_\nu=\kappa_\nu+\kappa_\nu^\text{sc} \label{eqn:abs-coeff}
		\end{align}
		where $\kappa_\nu$ is also known as the \textit{extinction coefficient} and $\kappa_\nu^\text{sc}$ the \textit{scattering coefficient}. The extinction coefficient is given as
		\begin{align}
			\kappa_\nu=\sum_{i}&\sum_{j>i}{[n_i-n_jG_{ij}(\nu)]\sigma_{ij}(\nu)} \nonumber\\
							&+\sum_{l}{n_en_l\sigma_{ll}(\nu,T)\left( 1-e^{-h\nu/kT} \right)}+\kappa_\nu^\text{addl} \label{eqn:ext-abs-coeff}
		\end{align}
		where the first term represents the contribution from \textit{bound-bound transitions} (i.e. absorption lines), the second represents that from \textit{bound-free transitions} (i.e. absorption edges) and the last term lumps together any additional opacity not written as detailed transitions. Here $G_{ij}=\frac{w_ig_i}{w_jg_j}$, with $g_i,g_j$ and $w_i,w_j$ being the respective degeneracies and statistical weights of levels $i$ and $j$ respectively. Also $\sigma_{ij}$ is the opacity of the transition between levels $i$ and $j$.
		
		The scattering coefficient in equation (\ref{eqn:abs-coeff}) is given as
		\begin{align}
			\kappa_\nu^\text{sc}=n_e\sigma_e + \sum_i{n_{\text{Ray},i}\sigma_{\text{Ray},i}} \label{eqn:sca-abs-coeff}
		\end{align}
		where $\sigma_e$, $\sigma_{\text{Ray},i}$ are the Thomson and Rayleigh scattering cross-sections respectively. For pure-hydrogen atmospheres with high temperatures, the Rayleigh scattering is negligible because the hydrogen atoms are expected to be fully ionized.
		
		In a similar manner, the \textit{total emission coefficient} is also expressed as a sum of thermal and scattering contributions as
		\begin{align}
			\eta_\nu^\text{total}=\eta_\nu+\eta_\nu^\text{sc} \label{eqn:em-coeff}
		\end{align}
		where
		\begin{align}
			\eta_\nu=\left(\dfrac{2h\nu^3}{c^2}\right)&\left[ \sum_{i}\sum_{j>i}{n_jG_{ij}(\nu)\sigma_{ij}(\nu)} \right. \nonumber\\
							&\left.+\sum_{l}{n_en_l\sigma_{ll}(\nu,T)e^{-h\nu/kT}}\right]+\eta_\nu^\text{addl} \label{eqn:ext-em-coeff}
		\end{align}
		in which, again, any additional emissivity is lumped together and given using the Planck function as $\eta_\nu^\text{addl}=\kappa_\nu^\text{addl}B_\nu$.
		
		Finally, taking into account the \textit{convective flux} $F_\text{conv}$ (if any) within the differential form of the radiative equilibrium equation (\ref{eqn:rad-equil}) as
		\begin{align}
			\integ{0}{\infty}{\dvt{(f_\nu J_\nu)}{\tau_\nu}}{\nu}+\dfrac{F_\text{conv}}{4\pi}=\dfrac{\sigma_R}{4\pi}T_\text{eff}^4 \label{eqn:rad-equil-2}
		\end{align}
		Differentiating equation (\ref{eqn:rad-equil-2}) with respect to the Lagrangian mass $m$ finally gives the \textit{radiative+convective equilibrium equation} as
		\begin{align}
			\integ{0}{\infty}{(\kappa_\nu J_\nu-\eta_\nu)}{\nu}+\dfrac{\rho}{4\pi}\dvt{F_\text{conv}}{m}=0 \label{eqn:rad-conv-equil}
		\end{align}
		
		\subsubsection{Numerical solution using ALI}
		\quad\\
		Accelerated lambda iteration (ALI) is a technique used to numerically solve the radiative transfer problem, particularly in the context of stellar atmosphere modeling \cite{hubeny2003accelerated}. It accelerates the computational process by introducing a correction term that takes into account the influence of previous iterations. A simplified procedural overview of this scheme is as follows:
		\begin{enumerate}
			\item \textit{Initial guess}: Start with an initial guess for the specific intensity of radiation at different frequencies within the atmosphere. \label{step:ali-init-guess}
			\item \textit{Lambda iteration}: Calculate how the radiation interacts with the atmosphere using the current guess for intensity, considering the absorption and emission by the species in the atmosphere. \label{step:ali-lmda-itr}
			\item \textit{Correction term}: Instead of simply accepting the result from step \ref{step:ali-lmda-itr}, ALI calculates a correction term based on the difference between the current guess and the result. \label{step:ali-corr-term}
			\item \textit{Update guess}: Add this correction term to the initial guess to obtain a new estimate for the intensity. \label{step:ali-update-guess}
			\item \textit{Iterate}: Repeat steps \ref{step:ali-lmda-itr}--\ref{step:ali-update-guess} until the solution converges, meaning the intensity values no longer change significantly between iterations. \label{step:ali-iterate}
		\end{enumerate}
		
		In order to numerically solve the structural equations for NLTE, they are first discretized over all three continuous variables: optical depth $\tau$, frequency $\nu$ and angle $\theta$. These discretized structural equations, together with a number of auxiliary relations, form a system of highly coupled, non-linear equations. Then the \textit{complete linearization scheme} \cite{auer1970non} is used to treat all equations on the same footing, thus solving all structural equations simultaneously. The T\"{u}bingen NLTE Model Atmosphere Package (TMAP)$^{26}$ code uses ALI to calculate the stellar atmosphere by taking into account metal line-blanketing. With an extended grid of such atmospheres of pure-hydrogen models, synthetic spectral were calculated and these are eventually used as table models in XSPEC in the present work to simulate the source emission.