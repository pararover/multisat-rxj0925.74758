\section{Data Reduction and Analysis} \label{sec:reduction-analysis}
    \begin{itemize}
        \item Will contain the details of the data reduction routines used
        \item Will contain the software tools used for data analysis
    \end{itemize}
    
    \subsection{ASCA SIS1 observations}
    
    \subsection{Chandra ACIS observations}
    
    \subsection{XMM-Newton EPIC-pn observations}
    	The SSS \source\ was observed by all the instruments, viz. EPIC-MOS 1, EPIC-MOS 2, EPIC-pn and RGS, on-board the XMM-Newton observatory for $\sim 52$ ks on 16 December 2000. Whereas, we had retrieved data from all the instruments, we decided to use only the EPIC-pn data. The reason for this being two-fold:
    	\begin{enumerate}[i.]
    		\item The spectral region of interest is of the lowest energies detectable by EPIC, and the pn detector has a comparatively higher sensitivity than the MOS detectors at lower energies \cite{stecchini2023revisiting,mateos2009statistical}.
    		\item Currently, the high resolution grating spectra (such as those produced by the RGS) yield unacceptable fits to atmosphere models of SSS. Also, no atmosphere model has yet been able to reproduce all the details in such grating spectra \cite{ness2020complications}.
    	\end{enumerate}
    As per recommendations by the XMM-Newton SOC, the data analysis was restricted to energies above 0.2 keV\footnote{\url{https://xmmweb.esac.esa.int/docs/documents/CAL-TN-0018.pdf}}. The data reduction procedures were performed using the \textit{XMM-Newton Science Analysis System} (SAS) version 21.0.0.
    
    The Observation Data Files (ODF) were downloaded using the online archival query interface at the \textit{High Energy Astrophysics Science Archive Research Center} (HEASARC)\footnote{\url{https://heasarc.gsfc.nasa.gov/db-perl/W3Browse/w3browse.pl}}.
    \subsection{NICER XTI observations}
    
