\section{Introduction}
    %To include the following
    %\begin{itemize}
        %\item Current literature on SSS
        %\item Current literature specifically on MR Vel -- focus on the lack of proper spectral analysis till date
        %\item Novelty of our approach -- elaborate on the reason for choosing multi-satellite data
        %\item Elaborate on the objective and what has been obtained in our work
    %\end{itemize}
    
    % Modified following reviewer comments 1.1 and 1.2:
    Supersoft X-ray sources (hereafter SSS) represent an important class of celestial objects. They were initially noticed in the soft X-ray survey  of the Large Magellanic Cloud galaxy (LMC) using the \textit{Einstein Observatory} by Long \textit{et al.} (1981) \cite{long81}. These objects were later recognized as a distinct class of intrinsically luminous X-ray sources by Trümper \textit{et al.} (1991) \cite{trumper1991x}, and also Greiner \textit{et al.} (1991) \cite{greiner1991rosat}, following the observation of more SSS in the LMC using the \textit{Roentgen Observatory Satellite} (\textit{ROSAT}). Subsequently, Kahabka and van den Heuvel (1997) \cite{kahabka97} have tabulated other SSS identified using \textit{ROSAT} in the Milky Way galaxy (hereafter Galactic SSS), the Small Magellanic Cloud galaxy (SMC), the Andromeda galaxy (M31) and the Local Group galaxy (NGC 55), in addition to those detected in the LMC. Ever since, hundreds of objects exhibiting similar characteristics have been documented, many of which are considered candidates or confirmed members of this class, thereby significantly expanding the catalogue of SSS \cite{kahabkatrumper1996, steinerdiaz1998, greiner2000, pietsch2003deep, di2003luminous, orio2010census, henze2010recent, sturm2012new, galiullin2021populations}.
    
    There have been relatively fewer detections of Galactic SSS, as compared to other galaxies. This may be attributed to significant interstellar extinction of soft X-rays. Situated at the edge of the MW and along its Galactic plane, the soft X-ray photons emitted by Galactic SSS face considerable interstellar absorption. Suggestions were initially made that SSS within the Galactic plane are observable only if they are within ~1 kpc from us. %must be within a distance of approximately 1 kpc to remain observable.
    Beyond this distance, interstellar absorption is exected to become sufficiently pronounced to render them undetectable \cite{van1992accreting}. But later discoveries indicated the presence of Galactic SSS beyond 1 kpc as well.
    
    SSS are classified based on their extremely soft X-ray spectra, which typically peak in the range $20-100$ eV, corresponding to effective blackbody temperatures of the order $\sim 10^5$ to $\sim 10^6$ K, while their luminosities can be typically as high as the Eddington limit ($\sim 10^{36}$ to $\sim 10^{38}$ erg s$^{-1}$) \cite{kahabka06}. A thorough understanding of the spectral properties of SSS, which encompass their effective temperatures, luminosities, and the complex absorption and emission features within their spectra, is crucial for explaining the thermonuclear processes occurring on the surfaces of white dwarfs. Such insights may better explain the late evolutionary stages of compact objects in binary systems, particularly in investigating the role of SSS as being potential progenitors of Type Ia supernovae \cite{di2006luminous}.
    
    % Supersoft X-ray sources (SSS) represent an important class of celestial objects. They were initially recognized as a distinct class of intrinsically luminous X-ray sources by Trümper \textit{et al.} \cite{trumper1991x}, Greiner \textit{et al.} \cite{greiner1991rosat}, and Kahabka \textit{et al.} \cite{kahabka97}. These sources are classified based on their X-ray luminosities, which are typically around the Eddington limit ($\sim 10^{38}$ erg s$^{-1}$), indicating their exceptionally high brightness. However, their defining feature is their extremely soft X-ray spectra, with weak or negligible emission beyond $\sim 1$ keV and effective blackbody temperature no more than $\sim 100$ eV (or $\lesssim 1200$ kK) \cite{kahabka06}. The understanding of the spectral characteristics of SSS promises to pave the way for a deeper exploration of their role in the broader astrophysical landscape.
    
    % SSS were initially observed by the Einstein Observatory during a soft X-ray survey of the Large Magellanic Cloud (LMC) by Long et al. in 1981 \cite{long81}. Over time, hundreds of objects exhibiting similar characteristics have been documented, many of which are considered candidates or confirmed members of this class. The catalog of SSS has expanded significantly, now encompassing sources not only within the LMC but also various other celestial bodies, including the Milky Way (MW), Small Magellanic Cloud (SMC), and Andromeda (M31) and numerous other galaxies \cite{kahabkatrumper1996,steinerdiaz1998,greiner2000,pietsch2003deep,di2003luminous,orio2010census,henze2010recent,sturm2012new,galiullin2021populations}.
    
    %SSS are classified based on their X-ray luminosities, which are typically around the Eddington limit ($\sim 10^{38}$ erg s$^{-1}$), indicating their exceptionally high brightness. However, their defining feature is their extremely soft X-ray spectra, with weak or negligible emission beyond $\sim 1$ keV and effective blackbody temperature no more than $\sim 100$ eV (or $\lesssim 1200$ kK) \cite{kahabka06}. The understanding of the spectral characteristics of SSS promises to pave the way for a deeper exploration of their role in the broader astrophysical landscape.
    
    The optical spectra of the SSS in the LMC and SMC exhibit similarities with those of low-mass X-ray binaries -- characterized by strong emission lines such as that of He II at 4686 \AA\ and hydrogen Balmer lines. Subsequent numerical calculations suggested that emission from SSS could result from near-Eddington accretion onto neutron stars \cite{kylafis93}, although subsequent analyses favoured white dwarfs as the emitting objects for supersoft X-rays \cite{vandenHeuvel92}.
    
    Employing Stefan-Boltzmann's law with typical luminosity and effective temperature values for SSS, the estimated radius of the emitting object aligns with that of a white dwarf, supporting the hypothesis of accretion onto white dwarfs as the source of supersoft X-rays, akin to accreting neutron stars and black holes in classical X-ray binaries. Van den Heuvel \textit{et al.} (1992) \cite{van1992accreting} proposed that white dwarfs with masses in the range $0.7–1.4\,M_\odot$ and having mass accretion rates $\sim 1-5\times 10^{-7}\,M_\odot\text{ yr}^{-1}$ produce supersoft X-rays, assuming the mass-accretor as the white dwarf and the companion as a main-sequence or post-main-sequence star within specific mass ranges. %\cite{van1992accreting}.
    So the currently accepted model for emission from SSS is \textit{the nuclear burning of accreting matter on the surface of a white dwarf}.
    
    Studies by various groups have explored different types of nuclear burning due to mass accretion on white dwarfs, depending on the thermal history of the white dwarf and conditions required for nuclear ignition, typically involving critical envelope masses %$\Delta M_\text{crit}$
    sustaining high temperatures ($\sim 10^8$ K) and pressures ($\gtrsim 10^{18}-10^{20}$ g cm$^{-1}$ s$^{-2}$) for nuclear burning via the CNO cycle \cite{paczynski78,prialnik78,sion79,sienkiewicz80,nomoto82,fujimoto82a,fujimoto82b,iben82,prialnik95,macdonald83}. The steady state accretion rate, crucial for understanding the relationship between accretion and nuclear burning, has been investigated in early calculations by Paczy\'{n}ski and Rudak (1980) \cite{paczynski80} and Iben (1982) \cite{iben82}, providing insights into the dynamics of hydrogen-rich matter accretion on white dwarfs. For a hydrogen-rich system, the steady state accretion rate was computed by Hachisu and Kato (2001) \cite{hachisu2001} to be
	\begin{align}
		\dot{M}_\text{steady}\sim 3.7\times 10^{-7}\left( \dfrac{M_\text{WD}}{M_\odot}-0.4 \right)\,M_\odot\,\text{yr}^{-1} \label{eqn:steady-mass-accr}
	\end{align}
	Nomoto (1982) \cite{nomoto82} identifies three potential regimes of nuclear burning on an accreting white dwarf that can lead to SSS emission. In the steady burning regime, accreted hydrogen undergoes stable nuclear burning on the white dwarf's surface without causing an expansion of its radius, provided the mass accretion rate remains within the range of $1-4\times 10^{-7}\,M_\odot\,\text{yr}^{-1}$. Outside this range, nuclear burning either occurs episodically in flashes at lower accretion rates, or proceeds steadily in a thin shell around the white dwarf at higher accretion rates. %Detailed calculations by Nomoto (1982) \cite{nomoto82}, suggest three possible regimes for nuclear burning on a white dwarf leading to SSS emission, wherein the steady burning of accreted hydrogen takes place on the surface of the white dwarf without radius expansion of the white dwarf when the mass accretion rate remains steadily confined to the range $1-4\times 10^{-7}\,M_\odot\,\text{yr}^{-1}$. Below or above this range, the accreted matter either burns in flashes or continues burning within a thin shell around the white dwarf respectively.
	
	\subsection{\source: A Luminous Galactic SSS}
	The particular luminous Galactic SSS known as MR Vel, and referred to as \source, was discovered by Motch \textit{et al.} (1994) \cite{motch1994} in the ROSAT Galactic Plane Survey (RGPS), whose scope is restricted to that region of the ROSAT All Sky Survey \cite{voges1993rosat} with absolute Galactic latitude $|b|\leqslant 20\degree$. In the J2000 frame, the right ascension and declination of \source\ are respectively $141.44042$ and $-47.96972$ ($\alpha=09$ $25$ $46.00$, $\delta=-47$ $58$ $17.4$, as resolved by SIMBAD$^1$).
	
	\source\ was the brightest SSS candidate source in RGPS, with ROSAT PSPC hardness ratios of $HR1=0.96\pm 0.03$ and $HR2=-0.69\pm 0.03$. It ought to be noted here that the hardness ratios serve as diagnostic tools for evaluating the spectral characteristics of X-ray sources. $HR1$ compares the relative contributions of medium-energy (0.40--2.40 keV) to low-energy (0.07--0.40 keV) photons, thereby providing insight into the overall spectral softness or hardness. On the other hand, $HR2$ quantifies the relative intensity of high-energy (1.00--2.40 keV) versus medium-energy (0.40--1.00 keV) photons, aiding in the assessment of the source's energy distribution. The values of $HR1$ and $HR2$ quoted above indicate an X-ray spectrum predominantly confined to the 0.40--2.40 keV range, with significant soft X-ray emission within the 0.4--1.00 keV range. %The values of $HR1$ and $HR2$ quoted above indicate a predominantly hard X-ray spectrum in the 0.40--2.40 keV range, while the negative $HR2$ suggests a softer emission within the medium-to-high energy range, with the spectrum being dominated by photons below 1.00 keV.
	
	Fitting a blackbody distribution with $kT$=30--55 eV to the ROSAT observations, a hydrogen column density in the range $n_H=(1.4-3.7)\times 10^{22}$ cm$^{-2}$ was obtained by Motch \textit{et al.} (1994) \cite{motch1994}. There is a considerable amount of uncertainty in current literature about its distance, with estimates ranging from 1 kpc to 10 kpc. Based on Gaia$^2$ Data Release 3 measurements, one can find negligible parallax of 0.13 mas at a distance of about $\sim 2.6$ kpc. This value is consistent with the suggestions of a 1-2 kpc distance range, derived from bolometric luminosity calculations by Motch \textit{et al.} (1994) \cite{motch1994} which assumed blackbody emission. The discrepancy between these estimates and the observed parallax may be attributed to significant interstellar extinction within the Vela Sheet molecular cloud, located at a distance of 425 pc along the line of sight \cite{grabelsky1987}. %This seems to be in line with the suggestion that its distance may be expected to be around 1-2 kpc, as indicated by bolometric luminosities assuming blackbody emission, subsequently bulk of the absorption taking place within the Vela Sheet molecular cloud at a distance of 425 pc along the line of sight. % luminosity suggests that its distance is likely to be $>5$ kpc.
	
	One of the earliest spectral studies on \source\ was conducted by Hartmann \textit{et al.} (1997) \cite{hartmann1999constraining} from BeppoSAX observations using the LECS instrument, wherein various model grids were used to fit the observed data. The best-fit model consisted of two components: a non-local thermodynamic equilibrium (NLTE) component and a collisional ionization equilibrium (CIE) component. The NLTE component accounts for deviations from thermodynamic equilibrium in the high-energy environment of white dwarfs, while the CIE component represents X-ray emission from a plasma through the white dwarf atmosphere. %The models that were applied were non-local thermodynamic equilibrium (NLTE) or Non-LTE models including metal line opacities. The high-energy environment around a white dwarf doesn’t allow for equilibrium. Non-LTE models account for this by allowing the populations of energy levels in atoms and ions to deviate from equilibrium, leading to more accurate simulations of high-temperature objects like white dwarfs.
	A single-component model led to a significant discrepancy above 1.2 keV, but the inclusion of the low-temperature CIE component improved the fit. Hartmann \textit{et al.} (1997) \cite{hartmann1999constraining} concluded that a single-component model, such as a blackbody, is insufficient to account for both soft and hard X-ray photons, and that a multi-component model is more appropriate.
	%% It was found that when a model consisting of a single component was used, a large discrepancy is obtained above 1.2 keV. By including the low-temperature CIE component with the NLTE component, the fit was found to improve significantly. Hartmann et al. (1997) \cite{hartmann1999constraining} concluded that, in order to fit the X-ray continuum spectrum of \source\ a model consisting of a single component, e.g. a blackbody, is insufficient to simultaneously account for both soft and hard X-ray photons and, thereby, a multi-component model would be more appropriate.
	%They found that if a single model component is assumed for \source, a large discrepancy is observed between the model and data above 1.19 keV. The emission above $\sim 1.2$ kev (i.e. Ne IX edge) can be accounted for by adding another spectral model component, namely collisional ionization equilibrium.
	
	Higher resolution data obtained using the grating instruments on-board the Chandra and XMM-Newton observatories revealed complex structures in the spectra of \source. Such spectra show the presence of P Cygni profiles of Fe XVII and O VIII, which typically arise in a wind. Earlier, Bearda et al. (2002) and Motch et al. (2002) had come to the conclusion that the \source\ spectra, as observed by the Chandra HETGS and the XMM-Newton RGS, cannot be reproduced by LTE or NLTE model atmospheres \cite{beardaChandra2002AA,motchXmmNewton2002AA}, even though there is little clarity on the reason for this. Obtaining an acceptable fit for \source\ spectrum assumes crucial importance at this juncture. In the absence of a proper model describing the emission spectrum, it becomes impossible to calculate its parameters such as effective temperature and luminosity.
	
	In the present work, our primary objective was to identify a robust spectral model capable of providing an acceptable fit to the continuum spectrum of \source. To this end, we devised an approach wherein we analyse spectral data for \source\ obtained using multiple observatories. A motivation for harnessing such multi-observatory science data over an extended period was to mitigate potential biases associated with individual instruments and temporal variations in observational conditions, thereby enabling a systematic examination of the source's spectral characteristics, in an attempt to constrain the underlying physical processes driving the observed supersoft X-ray emission.
	
	The amalgamation of observations from diverse space observatories offered unique insights into the spectral evolution of the supersoft X-ray source over time, shedding light on its dynamic behavior and emission properties. Through rigorous spectral modeling and comparative analysis, we made an attempt to establish a robust framework for characterizing the continuum spectrum of the supersoft source, thereby enhancing our understanding of its astrophysical nature and evolutionary trends. A comprehensive investigation into recurrent SSS like \source\ is paramount due to the emerging recognition of SSS as progenitors of type Ia supernovae, which serve as crucial standard candles in cosmology. By delving deeper into the astrophysical mechanisms governing SSS, we can potentially enhance our ability to make precise predictions regarding the occurrence of type Ia supernovae and subsequently refine cosmological distance measurements. %Such a study of a recurrent SSS like \source, we believe, is essential because SSS are now considered to be progenitors of type Ia supernovae, which are standard candles. A deeper understanding of its astrophysics might allow us to make accurate predictions regarding their occurence, and subsequent calculations of cosmological distance measurements. %\textbf{Insert text on why study of \source\ is important -- SSS are considered to be progenitors of type Ia supernovae.}