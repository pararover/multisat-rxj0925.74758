\section{Introduction}
    %To include the following
    %\begin{itemize}
        %\item Current literature on SSS
        %\item Current literature specifically on MR Vel -- focus on the lack of proper spectral analysis till date
        %\item Novelty of our approach -- elaborate on the reason for choosing multi-satellite data
        %\item Elaborate on the objective and what has been obtained in our work
    %\end{itemize}
    
    Supersoft X-ray sources (SSS) represent an important class of celestial objects. They were initially recognized as a distinct class of intrinsically luminous X-ray sources by Trümper \textit{et al.} \cite{trumper1991x}, Greiner \textit{et al.} \cite{greiner1991rosat}, and Kahabka \textit{et al.} \cite{kahabka97}. These sources are classified based on their X-ray luminosities, which are typically around the Eddington limit ($\sim 10^{38}$ erg s$^{-1}$), indicating their exceptionally high brightness. However, their defining feature is their extremely soft X-ray spectra, with weak or negligible emission beyond $\sim 1$ keV and effective blackbody temperature no more than $\sim 100$ eV (or $\lesssim 1200$ kK) \cite{kahabka06}. The understanding of the spectral characteristics of SSS promises to pave the way for a deeper exploration of their role in the broader astrophysical landscape.
    
    SSS were initially observed by the Einstein Observatory during a soft X-ray survey of the Large Magellanic Cloud (LMC) by Long et al. in 1981 \cite{long81}. Over time, hundreds of objects exhibiting similar characteristics have been documented, many of which are considered candidates or confirmed members of this class. The catalog of SSS has expanded significantly, now encompassing sources not only within the LMC but also various other celestial bodies, including the Milky Way (MW), Small Magellanic Cloud (SMC), and Andromeda (M31) and numerous other galaxies \cite{kahabkatrumper1996,steinerdiaz1998,greiner2000,pietsch2003deep,di2003luminous,orio2010census,henze2010recent,sturm2012new,galiullin2021populations}.
    
    The relatively fewer detections of SSS within our galaxy, compared to other galaxies, may be attributed to significant interstellar extinction of soft X-rays. Situated at the edge of the MW and along its galactic plane, soft X-rays emitted by galactic SSS face considerable interstellar absorption. Suggestions have been made that SSS within the galactic plane must be within a distance of approximately 1 kpc to remain observable. Beyond this distance, interstellar absorption becomes sufficiently pronounced to render them undetectable \cite{van1992accreting}.
    
    The optical spectra of the SSS in the Magellanic Cloud galaxies exhibit similarities with those of low-mass X-ray binaries -- characterized by strong emission lines such as He II at 4686 \AA\ and hydrogen Balmer lines. Subsequent numerical calculations suggested that SSS could result from near-Eddington accretion onto neutron stars \cite{kylafis93}, although subsequent analyses favoured white dwarfs as the emitting objects for supersoft X-rays \cite{vandenHeuvel92}.
    
    Employing Stefan-Boltzmann's law with typical luminosity and effective temperature values for SSS, the estimated radius of the emitting object aligns with that of a white dwarf, supporting the hypothesis of accretion onto white dwarfs as the source of supersoft X-rays, akin to accreting neutron stars and black holes in classical X-ray binaries. Van den Heuvel proposed that white dwarfs with masses in the range $0.7–1.4\,M_\odot$ and mass accretion rates $\sim 1-5\times 10^{-7}\,M_\odot\text{ yr}^{-1}$ produce supersoft X-rays, assuming the mass-accretor as the white dwarf and the companion as a main-sequence or post-main-sequence star within specific mass ranges \cite{van1992accreting}. Studies by various groups have explored different types of nuclear burning due to mass accretion on white dwarfs, depending on the thermal history of the white dwarf and conditions required for nuclear ignition, typically involving critical envelope masses %$\Delta M_\text{crit}$
    sustaining high temperatures ($\sim 10^8$ K) and pressures ($\gtrsim 10^{18}-10^{20}$ g cm$^{-1}$ s$^{-2}$) for nuclear burning via the CNO cycle \cite{paczynski78,prialnik78,sion79,sienkiewicz80,nomoto82,fujimoto82a,fujimoto82b,iben82,prialnik95,macdonald83}. The steady state accretion rate, crucial for understanding the relationship between accretion and nuclear burning, has been investigated in early calculations \cite{paczynski80,iben82}, providing insights into the dynamics of hydrogen-rich matter accretion on white dwarfs. For a hydrogen-rich system, the steady state accretion rate was given to be \cite{hachisu2001}
	\begin{align}
		\dot{M}_\text{steady}\sim 3.7\times 10^{-7}\left( \dfrac{M_\text{WD}}{M_\odot}-0.4 \right)\,M_\odot\,\text{yr}^{-1} \label{eqn:steady-mass-accr}
	\end{align}
	
	The particular luminous galactic SSS known as MR Vel, and referred to as \source, was discovered by Motch \textit{et al.} \cite{motch1994} in the ROSAT Galactic Plane Survey (RGPS), which  is defined as the $|b|\leqslant 20\degree$ region of the ROSAT All Sky Survey. In the J2000 frame, the right ascension and declination of \source\ is 141.44042 and -47.96972 ($\alpha$=09 25 46.00, $\delta$=-47 58 17.4: as resolved by Simbad\footnote{\url{http://simbad.u-strasbg.fr/simbad/}}).
	
	\source\ was the brightest SSS candidate source in RGPS, with ROSAT PSPC hardness ratios of $HR1=0.96\pm 0.03$ and $HR2=-0.69\pm 0.03$. Fitting a blackbody to the ROSAT observations, a hydrogen column density in the range $n_H=(1.4-3.7)\times 10^{22}$ cm$^{-2}$ can be obtained. There is a considerable amount of uncertainty in current literature about its distance, with estimates ranging from 1 kpc to 10 kpc. Consulting Gaia Data Release 3\footnote{\url{https://www.cosmos.esa.int/web/gaia/data-release-3}}, one can find negligible parallax. This fact along with its high luminosity suggests that its distance is likely to be $>5$ kpc.
	
	Hartmann \textit{et al.} applied non-local thermodynamic equilibrium (NLTE) models, which included metal line opacities, to the spectrum extracted from the observations by BeppoSAX LECS of RX J0925 on January 25-26 1997 \cite{hartmann1999constraining}. They found that if a single model component is assumed for \source, a large discrepancy is observed between the model and data above 1.19 keV. The emission above $\sim 1.2$ kev (i.e. Ne IX edge) can be accounted for by adding another spectral model component, namely collisional ionization equilibrium.
	
	Higher resolution data obtained using the grating instruments on-board the Chandra and XMM-Newton observatories revealed complex structures in the spectra of \source. Such spectra show the presence of P Cygni profiles of Fe XVII and O VIII, which typically arise in a wind. Earlier, Bearda et al. (2002) and Motch et al. (2002) had come to the conclusion that the \source\ spectra, as observed by the Chandra HETGS and the XMM-Newton RGS, cannot be reproduced by LTE or NLTE model atmospheres \cite{beardaChandra2002AA,motchXmmNewton2002AA}, even though there is little clarity on the reason for this. Obtaining an acceptable fit for \source\ spectrum assumes crucial importance at this juncture. In the absence of a proper model describing the emission spectrum, it becomes impossible to calculate its parameters such as effective temperature and luminosity.
	
	In the present work, our primary objective was to identify a robust spectral model capable of providing an acceptable fit to the continuum spectrum of \source. To this end, we devised an approach wherein we analyse spectral data for \source\ obtained using multiple observatories. A motivation for harnessing such multi-observatory science data over an extended period was to mitigate potential biases associated with individual instruments and temporal variations in observational conditions, thereby enabling a systematic examination of the source's spectral characteristics, in an attempt to constrain the underlying physical processes driving the observed supersoft X-ray emission.
	
	The amalgamation of observations from diverse space observatories offered unique insights into the spectral evolution of the supersoft X-ray source over time, shedding light on its dynamic behavior and emission properties. Through rigorous spectral modeling and comparative analysis, we made an attempt to establish a robust framework for characterizing the continuum spectrum of the supersoft source, thereby enhancing our understanding of its astrophysical nature and evolutionary trends. A comprehensive investigation into recurrent SSS like \source\ is paramount due to the emerging recognition of SSS as progenitors of type Ia supernovae, which serve as crucial standard candles in cosmology. By delving deeper into the astrophysical mechanisms governing SSS, we can potentially enhance our ability to make precise predictions regarding the occurrence of type Ia supernovae and subsequently refine cosmological distance measurements. %Such a study of a recurrent SSS like \source, we believe, is essential because SSS are now considered to be progenitors of type Ia supernovae, which are standard candles. A deeper understanding of its astrophysics might allow us to make accurate predictions regarding their occurence, and subsequent calculations of cosmological distance measurements. %\textbf{Insert text on why study of \source\ is important -- SSS are considered to be progenitors of type Ia supernovae.}