\section{Introduction}
    To include the following
    \begin{itemize}
        %\item Current literature on SSS
        \item Current literature specifically on MR Vel -- focus on the lack of proper spectral analysis till date
        \item Novelty of our approach -- elaborate on the reason for choosing multi-satellite data
        \item Elaborate on the objective and what has been obtained in our work
    \end{itemize}
    
    Supersoft X-ray sources (SSS) represent an important class of celestial objects. They were initially recognized as a distinct class of intrinsically luminous X-ray sources by Trümper \textit{et al.} \cite{trumper1991x}, Greiner \textit{et al.} \cite{greiner1991rosat}, and Kahabka \textit{et al.} \cite{kahabka97}. These sources are classified based on their X-ray luminosities, which are typically around the Eddington limit ($\sim 10^{38}$ erg s$^{-1}$), indicating their exceptionally high brightness. However, their defining feature is their extremely soft X-ray spectra, with weak or negligible emission beyond $\sim 1$ keV and effective blackbody temperature no more than $\sim 100$ eV (or $\lesssim 1200$ kK) \cite{kahabka06}. The understanding of the spectral characteristics of SSS promises to pave the way for a deeper exploration of their role in the broader astrophysical landscape.
    
    SSS were initially observed by the Einstein Observatory during a soft X-ray survey of the Large Magellanic Cloud (LMC) by Long et al. in 1981 \cite{long81}. Over time, hundreds of objects exhibiting similar characteristics have been documented, many of which are considered candidates or confirmed members of this class. The catalog of SSS has expanded significantly, now encompassing sources not only within the LMC but also various other celestial bodies, including the Milky Way (MW), Small Magellanic Cloud (SMC), and Andromeda (M31) and numerous other galaxies \cite{kahabkatrumper1996,steinerdiaz1998,greiner2000,pietsch2003deep,di2003luminous,orio2010census,henze2010recent,sturm2012new,galiullin2021populations}.
    
    The relatively fewer detections of SSS within our galaxy, compared to other galaxies, may be attributed to significant interstellar extinction of soft X-rays. Situated at the edge of the MW and along its galactic plane, soft X-rays emitted by galactic SSS face considerable interstellar absorption. Suggestions have been made that SSS within the galactic plane must be within a distance of approximately 1 kpc to remain observable. Beyond this distance, interstellar absorption becomes sufficiently pronounced to render them undetectable \cite{van1992accreting}.
    
    The optical spectra of the SSS in the Magellanic Cloud galaxies exhibit similarities with those of low-mass X-ray binaries -- characterized by strong emission lines such as He II at 4686 \AA\ and hydrogen Balmer lines. Subsequent numerical calculations suggested that SSS could result from near-Eddington accretion onto neutron stars \cite{kylafis93}, although subsequent analyses favoured white dwarfs as the emitting objects for supersoft X-rays \cite{vandenHeuvel92}.
    
    Employing Stefan-Boltzmann's law with typical luminosity and effective temperature values for SSS, the estimated radius of the emitting object aligns with that of a white dwarf, supporting the hypothesis of accretion onto white dwarfs as the source of supersoft X-rays, akin to accreting neutron stars and black holes in classical X-ray binaries. Van den Heuvel proposed that white dwarfs with masses in the range $0.7–1.4\,M_\odot$ and mass accretion rates $\sim 1-5\times 10^{-7}\,M_\odot\text{ yr}^{-1}$ produce supersoft X-rays, assuming the mass-accretor as the white dwarf and the companion as a main-sequence or post-main-sequence star within specific mass ranges. Studies by various groups have explored different types of nuclear burning due to mass accretion on white dwarfs, depending on the thermal history of the white dwarf and conditions required for nuclear ignition, typically involving critical envelope masses %$\Delta M_\text{crit}$
    sustaining high temperatures ($\sim 10^8$ K) and pressures ($\gtrsim 10^{18}–10^{20}$ g cm$^{-1}$ s$^{−2}$) for nuclear burning via the CNO cycle \cite{paczynski78,prialnik78,sion79,sienkiewicz80,nomoto82,fujimoto82a,fujimoto82b,iben82,prialnik95,macdonald83}. The steady state accretion rate, crucial for understanding the relationship between accretion and nuclear burning, has been investigated in early calculations \cite{paczynski80,iben82}, providing insights into the dynamics of hydrogen-rich matter accretion on white dwarfs. For a hydrogen-rich system, the steady state accretion rate was given to be \cite{hachisu2001}
	\begin{align}
		\dot{M}_\text{steady}\sim 3.7\times 10^{-7}\left( \dfrac{M_\text{WD}}{M_\odot}-0.4 \right)\,M_\odot\,\text{yr}^{-1} \label{eqn:steady-mass-accr}
	\end{align}